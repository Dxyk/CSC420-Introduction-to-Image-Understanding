% Template credit to University of Toronto CSC411 and CSC320
%------------------------------------------------------------------------------------
%	PACKAGES AND OTHER DOCUMENT CONFIGURATIONS
%------------------------------------------------------------------------------------

\documentclass{article}

\usepackage{fancyhdr} % Required for custom headers
\usepackage{lastpage} % Required to determine the last page for the footer
\usepackage{extramarks} % Required for headers and footers
\usepackage[usenames,dvipsnames]{color} % Required for custom colors
\usepackage{graphicx} % Required to insert images
\usepackage{subcaption}
\usepackage{listings} % Required for insertion of code
\usepackage{courier} % Required for the courier font
% Below are optional packages
\usepackage{amsmath}
\usepackage{amssymb}
\usepackage{float}
\usepackage{algorithm}
\usepackage[noend]{algpseudocode}


% Margins
\topmargin=-0.45in
\evensidemargin=0in
\oddsidemargin=0in
\textwidth=6.5in
\textheight=9.0in
\headsep=0.25in

\linespread{1.1} % Line spacing

% Set up the header and footer
\pagestyle{fancy}
\lhead{\hmwkAuthorName} % Top left header
\chead{\hmwkClass\ : \hmwkTitle} % Top center head
%\rhead{\firstxmark} % Top right header
\lfoot{\lastxmark} % Bottom left footer
\cfoot{} % Bottom center footer
\rfoot{Page\ \thepage\ of\ \protect\pageref{LastPage}} % Bottom right footer
\renewcommand\headrulewidth{0.4pt} % Size of the header rule
\renewcommand\footrulewidth{0.4pt} % Size of the footer rule

\setlength\parindent{0pt} % Removes all indentation from paragraphs


%------------------------------------------------------------------------------------
%	DOCUMENT STRUCTURE COMMANDS
%	Skip this unless you know what you're doing
%------------------------------------------------------------------------------------

% Header and footer for when a page split occurs within a problem environment
\newcommand{\enterProblemHeader}[1]{
	%\nobreak\extramarks{#1}{#1 continued on next page\ldots}\nobreak
	%\nobreak\extramarks{#1 (continued)}{#1 continued on next page\ldots}\nobreak
}

% Header and footer for when a page split occurs between problem environments
\newcommand{\exitProblemHeader}[1]{
	%\nobreak\extramarks{#1 (continued)}{#1 continued on next page\ldots}\nobreak
	%\nobreak\extramarks{#1}{}\nobreak
}

\setcounter{secnumdepth}{0} % Removes default section numbers
\newcounter{homeworkProblemCounter} % Creates a counter to keep track of the number of problems
\setcounter{homeworkProblemCounter}{0}

\newcommand{\homeworkProblemName}{}
\newenvironment{homeworkProblem}[1][Problem \arabic{homeworkProblemCounter}]{ % Makes a new environment called homeworkProblem which takes 1 argument (custom name) but the default is "Problem #"
	\stepcounter{homeworkProblemCounter} % Increase counter for number of problems
	\renewcommand{\homeworkProblemName}{#1} % Assign \homeworkProblemName the name of the problem
	\section{\homeworkProblemName} % Make a section in the document with the custom problem count
	\enterProblemHeader{\homeworkProblemName} % Header and footer within the environment
}{
	\exitProblemHeader{\homeworkProblemName} % Header and footer after the environment
}

\newcommand{\problemAnswer}[1]{ % Defines the problem answer command with the content as the only argument
	\noindent\framebox[\columnwidth][c]{\begin{minipage}{0.98\columnwidth}#1\end{minipage}} % Makes the box around the problem answer and puts the content inside
}

\newcommand{\homeworkSectionName}{}
\newenvironment{homeworkSection}[1]{ % New environment for sections within homework problems, takes 1 argument - the name of the section
	\renewcommand{\homeworkSectionName}{#1} % Assign \homeworkSectionName to the name of the section from the environment argument
	\subsection{\homeworkSectionName} % Make a subsection with the custom name of the subsection
	\enterProblemHeader{\homeworkProblemName\ [\homeworkSectionName]} % Header and footer within the environment
}{
	\enterProblemHeader{\homeworkProblemName} % Header and footer after the environment
}


%=================================================================

%------------------------------------------------------------------------------------
%	NAME AND CLASS SECTION
%------------------------------------------------------------------------------------

\newcommand{\hmwkTitle}{Assignment\ 2} % Assignment title
\newcommand{\hmwkClass}{CSC420} % Course/class
\newcommand{\hmwkAuthorName}{Xiangyu Kong, kongxi16} % Your name

%------------------------------------------------------------------------------------
%	TITLE PAGE
%------------------------------------------------------------------------------------

\title{
	\vspace{2in}
	\textmd{\textbf{\hmwkClass:\ \hmwkTitle}}\\
	%	\normalsize\vspace{0.1in}\small{Due\ on\ \hmwkDueDate}\\
	\vspace{0.1in}
	\vspace{3in}
}

\author{\textbf{\hmwkAuthorName}}

% Insert date here if you want it to appear below your name
\date{\today} 

%------------------------------------------------------------------------------------

\begin{document}
	
	\maketitle
	\clearpage
	
	%=========================================================
	%---------------------------------------------------------------------------------
	%	PROBLEM 1: 
	%---------------------------------------------------------------------------------
	%=========================================================
	\begin{homeworkProblem}
		
        See \textit{question\_1.py} for implementation.
        
		\begin{enumerate}
			
			\item 
			
			The result of applying linear interpolation to up-sample the image by 4 times are shown in Figure \ref{fig:1.1}.
            
            Figure \ref{fig:1.1.1} shows the first round of convolution along the first axis. 
            
            Figure \ref{fig:1.1.2} shows the second round of convolution along the second axis on top of the result of the first round.
            
            Comparing to the original image, the up-sampled image maintains most of the information because the images look alike.
            
            \begin{figure}[!htb]
                \begin{subfigure}{.3\textwidth}
                    \includegraphics[height=\linewidth, width=\linewidth]{images/question_1/1_1_1.jpg}
                    \centering
                    \caption{First round}
                    \label{fig:1.1.1}
                \end{subfigure}
                \begin{subfigure}{.3\textwidth}
                    \includegraphics[width=\linewidth]{images/question_1/1_1_2.jpg}
                    \centering
                    \caption{Second Round}
                    \label{fig:1.1.2}
                \end{subfigure}
                \begin{subfigure}{.3\textwidth}
                    \includegraphics[width=\linewidth]{images/orig/bee.jpg}
                    \centering
                    \caption{Original Image}
                    \label{fig:1.1.3}
                \end{subfigure}
                \centering
                \caption{}
                \label{fig:1.1}
            \end{figure}
        
            To achieve the same operation (up-sampling by a factor of 4), we can multiply the 1-dimensional filter by itself to get a 2-dimensional filter. This filter will be separable, and will produce the same result as convolving with the 1-dimensional filter twice on different directions.
            
            In this specific case of up-sampling by a factor of 4, we calculate that
            
            \begin{align*}
                h & = 
                \begin{bmatrix}
                0.25 & 0.5 & 0.75 & 1 & 0.75 & 0.5 & 0.25
                \end{bmatrix} \\
                h \times h^{T} & = 
                \begin{bmatrix}
                0.0625 & 0.125 & 0.1875 & 0.25 & 0.1875 & 0.125 & 0.0625 \\
                0.125 & 0.25 & 0.375 & 0.5 & 0.375 & 0.25 & 0.125 \\
                0.1875 & 0.375 & 0.5625 & 0.75 & 0.5625 & 0.375 &0.1875\\
                0.25 & 0.5 & 0.75 & 1 &	0.75 & 0.5 & 0.25\\
                0.1875 & 0.375 & 0.5625 & 0.75 & 0.5625 & 0.375 &0.1875\\
                0.125  & 0.25 & 0.375 &	0.5 & 0.375 & 0.25 & 0.125\\
                0.0625 & 0.125 & 0.1875 & 0.25 & 0.1875	& 0.125	& 0.0625
                \end{bmatrix} \\
            \end{align*}

            \newpage
            \item
            
            The generalized two dimensional linear interpolation reconstruction filter is 
            
            \begin{align*}
            h & = 
            \begin{bmatrix}
            \frac{1}{d} & \dots & \frac{d-1}{d} & 1 & \frac{d-1}{d} & \dots & \frac{1}{d}
            \end{bmatrix} \\
            h \times h^{T} & = 
            \begin{bmatrix}
            \frac{1}{d}^2 & \dots & \frac{d-1}{d} \times \frac{1}{d} & 1 \times \frac{1}{d} & \frac{d-1}{d} \times \frac{1}{d} & \dots & \frac{1}{d}^2 \\
            \frac{d-1}{d} \times \frac{1}{d} & \ddots & \dots & \dots & \dots & \dots & \vdots \\
            \vdots & \dots & \dots & 1 &	\dots & \dots & \vdots\\
            \frac{d-1}{d} \times \frac{1}{d} & \dots & \dots & \dots & \dots	& \dots	& \frac{d-1}{d}^2
            \end{bmatrix} \\
            \end{align*}
            
            Since this is a separable filter (composed of $h \times h^T$), it is equivalent of applying the 1D linear interpolation filter twice in both direction.
            
            The results of the 2D linear interpolation filter convolution is shown in Figure \ref{fig:1.2.1}. Comparing to the result of applying 1D linear interpolation filter twice in two directions (Figure \ref{fig:1.2.2}), the results do not have any difference. 
            
            \begin{figure}[!htb]
                \begin{subfigure}{.3\textwidth}
                    \includegraphics[width=\linewidth]{images/question_1/1_2.jpg}
                    \centering
                    \caption{2D Filter}
                    \label{fig:1.2.1}
                \end{subfigure}
                \begin{subfigure}{.3\textwidth}
                    \includegraphics[width=\linewidth]{images/question_1/1_1_2.jpg}
                    \centering
                    \caption{1D Filter}
                    \label{fig:1.1.2}
                \end{subfigure}
                \begin{subfigure}{.3\textwidth}
                    \includegraphics[width=\linewidth]{images/orig/bee.jpg}
                    \centering
                    \caption{Original Image}
                    \label{fig:1.2.2}
                \end{subfigure}
                \centering
                \caption{}
                \label{fig:1.2}
            \end{figure}
			
		\end{enumerate}
	
	\end{homeworkProblem}

    \clearpage
    %=========================================================
    %---------------------------------------------------------------------------------
    %	PROBLEM 2: 
    %---------------------------------------------------------------------------------
    %=========================================================
    \begin{homeworkProblem}
    	
    	\begin{enumerate}
    		
    		\item 
    		
    		The computational cost for $h * I$ when $h$ is not separable is $O(n^{2}m^{2})$.
    		
    		This is because each pixel in $I$ gets computed for $m^2$ times, and there are $n^2$ pixels in total.
    		
    		\item 
    		
    		The computational cost for $h * I$ when $h$ is separable is $O(m^{2}2n)$.
    		
    		This is because each pixel in $I$ gets computed for $2m$ times (only vertical and horizontal edge detecting vectors), and there are $n^2$ pixels in total
    		
    	\end{enumerate}
    	
    \end{homeworkProblem}
	
	
	%=========================================================
	%---------------------------------------------------------------------------------
	%	END
	%---------------------------------------------------------------------------------
	%=========================================================
	
\end{document}
