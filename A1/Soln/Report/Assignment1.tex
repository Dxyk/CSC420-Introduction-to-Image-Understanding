% Template credit to University of Toronto CSC411 and CSC320
%------------------------------------------------------------------------------------
%	PACKAGES AND OTHER DOCUMENT CONFIGURATIONS
%------------------------------------------------------------------------------------

\documentclass{article}

\usepackage{fancyhdr} % Required for custom headers
\usepackage{lastpage} % Required to determine the last page for the footer
\usepackage{extramarks} % Required for headers and footers
\usepackage[usenames,dvipsnames]{color} % Required for custom colors
\usepackage{graphicx} % Required to insert images
\usepackage{subcaption}
\usepackage{listings} % Required for insertion of code
\usepackage{courier} % Required for the courier font
% Below are optional packages
\usepackage{amsmath}
\usepackage{amssymb}
\usepackage{float}
\usepackage{algorithm}
\usepackage[noend]{algpseudocode}


% Margins
\topmargin=-0.45in
\evensidemargin=0in
\oddsidemargin=0in
\textwidth=6.5in
\textheight=9.0in
\headsep=0.25in

\linespread{1.1} % Line spacing

% Set up the header and footer
\pagestyle{fancy}
\lhead{\hmwkAuthorName} % Top left header
\chead{\hmwkClass\ : \hmwkTitle} % Top center head
%\rhead{\firstxmark} % Top right header
\lfoot{\lastxmark} % Bottom left footer
\cfoot{} % Bottom center footer
\rfoot{Page\ \thepage\ of\ \protect\pageref{LastPage}} % Bottom right footer
\renewcommand\headrulewidth{0.4pt} % Size of the header rule
\renewcommand\footrulewidth{0.4pt} % Size of the footer rule

\setlength\parindent{0pt} % Removes all indentation from paragraphs


%------------------------------------------------------------------------------------
%	DOCUMENT STRUCTURE COMMANDS
%	Skip this unless you know what you're doing
%------------------------------------------------------------------------------------

% Header and footer for when a page split occurs within a problem environment
\newcommand{\enterProblemHeader}[1]{
	%\nobreak\extramarks{#1}{#1 continued on next page\ldots}\nobreak
	%\nobreak\extramarks{#1 (continued)}{#1 continued on next page\ldots}\nobreak
}

% Header and footer for when a page split occurs between problem environments
\newcommand{\exitProblemHeader}[1]{
	%\nobreak\extramarks{#1 (continued)}{#1 continued on next page\ldots}\nobreak
	%\nobreak\extramarks{#1}{}\nobreak
}

\setcounter{secnumdepth}{0} % Removes default section numbers
\newcounter{homeworkProblemCounter} % Creates a counter to keep track of the number of problems
\setcounter{homeworkProblemCounter}{0}

\newcommand{\homeworkProblemName}{}
\newenvironment{homeworkProblem}[1][Problem \arabic{homeworkProblemCounter}]{ % Makes a new environment called homeworkProblem which takes 1 argument (custom name) but the default is "Problem #"
	\stepcounter{homeworkProblemCounter} % Increase counter for number of problems
	\renewcommand{\homeworkProblemName}{#1} % Assign \homeworkProblemName the name of the problem
	\section{\homeworkProblemName} % Make a section in the document with the custom problem count
	\enterProblemHeader{\homeworkProblemName} % Header and footer within the environment
}{
	\exitProblemHeader{\homeworkProblemName} % Header and footer after the environment
}

\newcommand{\problemAnswer}[1]{ % Defines the problem answer command with the content as the only argument
	\noindent\framebox[\columnwidth][c]{\begin{minipage}{0.98\columnwidth}#1\end{minipage}} % Makes the box around the problem answer and puts the content inside
}

\newcommand{\homeworkSectionName}{}
\newenvironment{homeworkSection}[1]{ % New environment for sections within homework problems, takes 1 argument - the name of the section
	\renewcommand{\homeworkSectionName}{#1} % Assign \homeworkSectionName to the name of the section from the environment argument
	\subsection{\homeworkSectionName} % Make a subsection with the custom name of the subsection
	\enterProblemHeader{\homeworkProblemName\ [\homeworkSectionName]} % Header and footer within the environment
}{
	\enterProblemHeader{\homeworkProblemName} % Header and footer after the environment
}


%=================================================================

%------------------------------------------------------------------------------------
%	NAME AND CLASS SECTION
%------------------------------------------------------------------------------------

\newcommand{\hmwkTitle}{Assignment\ 1} % Assignment title
\newcommand{\hmwkClass}{CSC420} % Course/class
\newcommand{\hmwkAuthorName}{Xiangyu Kong, kongxi16} % Your name

%------------------------------------------------------------------------------------
%	TITLE PAGE
%------------------------------------------------------------------------------------

\title{
	\vspace{2in}
	\textmd{\textbf{\hmwkClass:\ \hmwkTitle}}\\
	%	\normalsize\vspace{0.1in}\small{Due\ on\ \hmwkDueDate}\\
	\vspace{0.1in}
	\vspace{3in}
}

\author{\textbf{\hmwkAuthorName}}

% Insert date here if you want it to appear below your name
\date{\today} 

%------------------------------------------------------------------------------------

\begin{document}
	
	\maketitle
	\clearpage
	
	%---------------------------------------------------------------------------------
	%	PROBLEM 1
	%---------------------------------------------------------------------------------
	\begin{homeworkProblem}
		\begin{enumerate}
			\item 
			
			The computational cost for $h * I$ when $h$ is not separable is $O(n^{2}m^{2})$.
			
			This because each pixel in $I$ gets computed for $m^2$ time, and there are $n^2$ pixels in total.
			
			\item 
			
			The computational cost for $h * I$ when $h$ is not separable is $O(m^{2}2n)$.
		\end{enumerate}
	\end{homeworkProblem}
	
	%----------------------------------------------------------------------------------

	%---------------------------------------------------------------------------------
	%	PROBLEM 2: 
	%---------------------------------------------------------------------------------
	\begin{homeworkProblem}
		
		Canny Edge Detection Steps:
		\begin{enumerate}
			\item Filter the image with derivative of Gaussian in both horizontal and vertical directions.
			
			\begin{itemize}
				\item The purpose of this is to smooth the image and remove the noise.
				\item To do this, we apply Gaussian filter to convolve with the image.
			\end{itemize}
			
			\item Find the magnitude and direction for the gradients
			
			\begin{itemize}
				\item The purpose of this is to find the possible edges
				\item To do this, we apply edge detection filters (for example, Sobel) with different directions and convolve with the image.
			\end{itemize}
			
			\item Non-maximum suppression
			
			\begin{itemize}
				\item Get rid of the spurious response from edge detection produced by noise.
				\item To do this, we only take local maximum or minimum of the edges.
			\end{itemize}
		
			\item Linking and Thresholding
			
			\begin{itemize}
				\item The purpose of this is to connect the unlinked edges.
				\item To do this, define 2 thresholds low and high. We use the high threshold's results to start the edge curves and use low threshold's results to connect the unlinked edges.
			\end{itemize}
			
		\end{enumerate}
		
	\end{homeworkProblem}
	
	%----------------------------------------------------------------------------------

	%---------------------------------------------------------------------------------
	%	PROBLEM 3: 
	%---------------------------------------------------------------------------------
	\begin{homeworkProblem}
		Laplacian of Gaussian filters are symmetrical. It has the shape of \ref{fig:3}. We can see that there is a decrease and then an increase going towards the center. This activates the edges because edges are composed of sudden change of pixel intensities.
		
		\begin{figure}[!ht]
			\centering
			\includegraphics[width=.5\linewidth]{images/03_01.png}
			\caption{Laplacian of Gaussian with $\sigma = 2.5$}
			\label{fig:3}
		\end{figure}
	
	\end{homeworkProblem}
	\clearpage
	
	%---------------------------------------------------------------------------------
	%---------------------------------------------------------------------------------
	%	PROBLEM 4: 
	%---------------------------------------------------------------------------------
	\begin{homeworkProblem}
		
		See code implementations in question\_4.py.
		
		\begin{enumerate}
			\item 
			
			See sample results in \ref{fig:4.1}. The results are the grayscale image correlation and convolution with the sharpening filter 
			
			\[
			\begin{bmatrix}
				0 & 0 & 0 \\
				0 & 2 & 0  \\
				0 & 0 & 0
			\end{bmatrix} - \dfrac{1}{9}
			\begin{bmatrix}
			1 & 1 & 1 \\
			1 & 1 & 1  \\
			1 & 1 & 1
			\end{bmatrix}
			\]
			
			\begin{figure}[!htb]
				\minipage{0.3\textwidth}
					\includegraphics[width=\linewidth]{images/4_correlation_full.jpg}
					\caption{Result of my\_correlation}
					\label{fig:4.1.1}
				\endminipage\hfill
				\minipage{0.3\textwidth}
					\includegraphics[width=\linewidth]{images/gray.jpg}
					\caption{Original image}
					\label{fig:4.1.2}
				\endminipage\hfill
				\minipage{0.3\textwidth}
					\includegraphics[width=\linewidth]{images/4_convolution_full.jpg}
					\caption{Result of my\_convolution}
					\label{fig:4.1.3}
				\endminipage\hfill
			\end{figure}
		
		
			\item 
			
		
		\end{enumerate}
	\end{homeworkProblem}
	\clearpage
	
	%---------------------------------------------------------------------------------
	%---------------------------------------------------------------------------------
	%---------------------------------------------------------------------------------
	%	PROBLEM 5:
	%---------------------------------------------------------------------------------
	\begin{homeworkProblem}
		
		See code implementations in question\_5.py.
		
		\begin{enumerate}
			\item 
			
			Separable Filters allow correlation and convolution operations to be performed at $2K$ operation cost instead of $K^2$. This is because separable filters are able to be split into $1D$ horizontal and $1D$ vertical filters. The result of convolving the image with the 2 $1D$ filters is the same as convolving the image with the original separable filter.
			
			This is achieved when when the Singular Value Decomposition has only one non-zero single value.
			
			$$
			F = U \Sigma V^{T} = \Sigma^{k}_{i = 0} \sigma_{i} u_{i} v_{i}^{T}
			$$
			
			where $\sqrt{\sigma_{1}} \boldsymbol{u}_{1}$ and $\sqrt{\sigma_{1}} \boldsymbol{v}_{1}$ are the vertical and horizontal filters.
			
			\item 
			
			The Separable filter is
			$
			\dfrac{1}{16}
			\begin{bmatrix}
			1 & 2 & 1 \\
			2 & 4 & 2 \\
			1 & 2 & 1
			\end{bmatrix}
			$.
			
			The output for the separable filter is 
			
			\begin{lstlisting}
	[[-0.25]
	[-0.5 ]
	[-0.25]]
	
	[[-0.25 -0.5  -0.25]]
	True
			\end{lstlisting}
			
			The Inseparable filter is 
			$
			\begin{bmatrix}
			0 & -1 & 0 \\
			-1 & 5 & -1 \\
			0 & -1 & 0
			\end{bmatrix}
			$.
			
			The output for the separable filter is 
			
			\begin{lstlisting}
	False
			\end{lstlisting}
			
		\end{enumerate}
	\end{homeworkProblem}
	\clearpage
	
	%---------------------------------------------------------------------------------
	
	
		%---------------------------------------------------------------------------------
	%---------------------------------------------------------------------------------
	%---------------------------------------------------------------------------------
	%	PROBLEM 6:
	%---------------------------------------------------------------------------------
	\begin{homeworkProblem}

		See code implementations in question\_6.py.

		\begin{enumerate}
			%a
			\item 
			
			The image with noise is as follows
			
			
			\begin{figure}[!htb]
				\minipage{0.45\textwidth}
				\includegraphics[width=\linewidth]{images/6_a_rand_noise.jpg}
				\caption{Result of add\_rand\_correlation}
				\label{fig:6.1.1}
				\endminipage\hfill
				\minipage{0.45\textwidth}
				\includegraphics[width=\linewidth]{images/gray.jpg}
				\caption{Original image}
				\label{fig:6.1.2}
				\endminipage\hfill
			\end{figure}
		
			% b
			\item 
			
			The recovered image is as follows. The chosen filter was the mean filter
			$
			\dfrac{1}{9}
			\begin{bmatrix}
			1 & 1 & 1 \\
			1 & 1 & 1 \\
			1 & 1 & 1
			\end{bmatrix}
			$. This was chosen because averaging the pixels will reduce the effect of the noise and make the noisy pixel resemble the average of its neighbors.
			
			\begin{figure}[!htb]
				\minipage{0.3\textwidth}
				\includegraphics[width=\linewidth]{images/6_a_rand_noise.jpg}
				\caption{add\_rand\_correlation}
				\label{fig:6.2.1}
				\endminipage\hfill
				\minipage{0.3\textwidth}
				\includegraphics[width=\linewidth]{images/6_b_mean.jpg}
				\caption{The recovered image}
				\label{fig:6.2.2}
				\endminipage\hfill
				\minipage{0.3\textwidth}
				\includegraphics[width=\linewidth]{images/gray.jpg}
				\caption{The original image}
				\label{fig:6.2.3}
				\endminipage\hfill
			\end{figure}
			

			% c
			\newpage
			\item 
			
			The image with salt and pepper noise is as bellow:
			
			\begin{figure}[!htb]
				\minipage{0.45\textwidth}
				\includegraphics[width=\linewidth]{images/6_c_sp_noise.jpg}
				\caption{add\_salt\_and\_pepper}
				\label{fig:6.4.1}
				\endminipage\hfill
				\minipage{0.45\textwidth}
				\includegraphics[width=\linewidth]{images/gray.jpg}
				\caption{The original image}
				\label{fig:6.4.2}
				\endminipage\hfill
			\end{figure}
		
			% d
			\item 
			
			Trying to recover salt and pepper noisy image does not yield a very good result: the output image is still noisy. The chosen filter is the median filter. This is achieved with OpenCV's $medianBlur$ method. This works because salt and pepper noise is very extreme. The pixel has signal of either 1 or 0 (white or black). This affects the mean vastly, where as it does not really affect the median of the nearby pixels.
			
			\begin{figure}[!htb]
				\minipage{0.3\textwidth}
				\includegraphics[width=\linewidth]{images/6_d_mean.jpg}
				\caption{mean filter}
				\label{fig:6.5.1}
				\endminipage\hfill
				\minipage{0.3\textwidth}
				\includegraphics[width=\linewidth]{images/6_d_median.jpg}
				\caption{median filter}
				\label{fig:6.5.2}
				\endminipage\hfill
				\minipage{0.3\textwidth}
				\includegraphics[width=\linewidth]{images/gray.jpg}
				\caption{The original image}
				\label{fig:6.5.3}
				\endminipage\hfill
			\end{figure}
			
			% e
			\newpage
			\item 
			
			
		\end{enumerate}
	\end{homeworkProblem}
	\clearpage
	
	%---------------------------------------------------------------------------------
	
\end{document}
